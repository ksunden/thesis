\chapter{\texttt{yaq}} \label{cha:yaq}

\clearpage

\section{Introduction}  % =========================================================================

\clearpage

\section{The \yaq Protocol}  % =========================================================================

% The yaq paper here

\subsection{Properties}

\subsection{Traits}

\clearpage

\section{Tools}

\subsection{\texttt{yaq-traits}}

\hypertarget{installation}{%
\subsubsection{installation}\label{installation}}

yaq-traits can be installed via
\href{https://pypi.org/project/yaq-traits/}{PyPI} or
\href{https://anaconda.org/conda-forge/yaq-traits}{conda-forge}.

\begin{codefragment}{bash}
$ pip install yaq-traits
\end{codefragment}

\begin{codefragment}{bash}
$ conda config --add channels conda-forge
$ conda install yaq-traits
\end{codefragment}

\hypertarget{definitions}{%
\subsubsection{definitions}\label{definitions}}

\begin{description}
\tightlist
\item[\href{https://avro.apache.org/docs/current/spec.html\#Protocol+Declaration}{Avro
Protocol (AVPR)}]
The Avro protocol is the fully specified description of the daemon. It
describes the exposed method signatures (names, arguments, defaults to
arguments, text description). In addition, yaq avpr files include
information about the configuration, state, and traits of a daemon. Avpr
is a JSON file.
\item[\href{https://github.com/toml-lang/toml}{TOML}]
TOML is a configuration file. Here, we use TOML files to specify the
minimum info about a daemon. (i.e. behavior and descriptions inherited
via traits is \emph{not} included in the TOML) A full reference is
available below.
\item[\href{https://yaq.fyi/traits/}{Trait}]
A trait is simply a collection of exposed methods, configuration, and
state used to encourage shared behavior among similar yaq daemons with
different implementations
\end{description}

\hypertarget{usage}{%
\subsubsection{usage}\label{usage}}

yaq-traits is a command line application.

\hypertarget{help}{%
\paragraph{help}\label{help}}

Help: learn more, right from your terminal.

\begin{codefragment}{bash}
$ yaq-traits --help
Usage: yaq-traits [OPTIONS] COMMAND [ARGS]...

Options:
  --version  Show the version and exit.
  --help     Show this message and exit.

Commands:
  check
  compose
\end{codefragment}

Try \texttt{yaq-traits\ \ -\/-help} to learn more about a particular
command.

\hypertarget{list}{%
\paragraph{list}\label{list}}

List: list available traits

\begin{codefragment}{bash}
$ yaq-traits list
has-limits
has-measure-trigger
has-position
has-turret
is-daemon
is-discrete
is-homeable
is-sensor
uses-i2c
uses-serial
uses-uart
\end{codefragment}

\hypertarget{compose}{%
\paragraph{compose}\label{compose}}

Compose: Convert a simplified TOML file to a fully specified AVPR. This
takes a path to a TOML file as an argument, and prints the AVPR to
standard out.

\begin{codefragment}{bash}
$ yaq-traits compose my-daemon.toml > my-daemon.avpr
\end{codefragment}

\hypertarget{get}{%
\paragraph{get}\label{get}}

Get: Retrieve a fully specified AVPR for a trait (similar to compose,
but not a full daemon). This takes a name of a trait as an argument, and
prints the AVPR to standard out.

\begin{codefragment}{bash}
$ yaq-traits get has-limits > has-limits.avpr
\end{codefragment}

\hypertarget{check}{%
\paragraph{check}\label{check}}

Check: Verify that an AVPR file matches current trait behavior. This
takes a path to an AVPR file as an argument, and prints a table of
traits and whether the trait was found to be accurate. If it fails, the
traits which are not specified are explicitly printed and a nonzero exit
status is returned. The primary intent of \texttt{check} is to be used
by Continuous Integration tests, which consistently get the latest
version and will alert you if there is a change to the traits.

\begin{codefragment}{bash}
$ yaq-traits check fake-continuous-hardware.avpr
+---------------------+----------+----------+
| trait               | expected | measured |
+---------------------+----------+----------+
| has-limits          | true     | true     |
| has-position        | true     | true     |
| has-turret          | false    | false    |
| is-daemon           | true     | false    |
| is-discrete         | false    | false    |
| is-homeable         | false    | false    |
| uses-i2c            | false    | false    |
| uses-serial         | false    | false    |
| uses-uart           | false    | false    |
| has-measure-trigger | false    | false    |
| is-sensor           | false    | false    |
+---------------------+----------+----------+
Error: failed to verify expected trait(s):
  is-daemon
\end{codefragment}

What to do when a check fails:

\begin{itemize}
\tightlist
\item
  check that you have the most up to date \texttt{yaq-traits}
\item
  check the
  \href{https://github.com/yaq-project/yaq-traits/blob/main/CHANGELOG.md}{changelog}
  for \texttt{yaq-traits} to see what has been changed recently
\item
  In many cases, the accompanying behavior change will be implemented in
  the \texttt{core} package for your language (e.g.
  \href{https://github.com/yaq-project/yaq-python/blob/main/yaqd-core/CHANGELOG.md}{Python}).
  If so, all you need to do is pin the version of core and rerun
  \texttt{yaq-traits\ compose}
\item
  if the change requires your attention, make the necessary code changes
  and then rerun \texttt{yaq-traits\ compose}
\item
  feel free to reach out to the \href{https://yaq.fyi/contact/}{yaq
  developers} or
  \href{https://github.com/yaq-project/yaq-traits/issues}{raise an
  issue} if you are unsure of what is needed
\end{itemize}

\hypertarget{yaq-toml-file-specification}{%
\subsubsection{yaq TOML file
specification}\label{yaq-toml-file-specification}}

In general, the TOML file is used to specify the minimal description of
a daemon. Inherited behavior from traits are not included. You may
assign or reassign the default value of the default, but should not
change the documentation, type, or arguments to a message. The same
format is used to specify traits within `yaq-traits`.

\hypertarget{frontmatter}{%
\paragraph{frontmatter}\label{frontmatter}}

These identifying information about the daemon are provided as top level
keys.

\begin{description}
\tightlist
\item[\textbf{protocol} (string)]
The name of the daemon. The equivalent for a trait definition is
\textbf{trait}.
\item[\textbf{doc} (string)]
A description of the daemon, rendered at the top of its documentation
page.
\item[\textbf{traits} (\{'items': 'string', 'type': 'array'\})]
List of traits implemented by the daemon. In trait definitions, the
analogous entry is \textbf{requires}, which lists traits that are
implied by using that trait. If a traits is implied by other traits
(e.g. \texttt{has-limits} requires \texttt{has-position}) the required
trait need not be specified. \texttt{is-daemon} must be specified by all
daemons.
\item[\textbf{hardware} (\{'items': 'string', 'type': 'array'\})]
A list of supported hardware, formatted as 'make:model'. Supported
hardware should also appear in
\href{https://github.com/yaq-project/yaq-fyi/blob/main/known-hardware.toml}{known-hardware}.
Used only for building documentation.
\end{description}

\hypertarget{links}{%
\paragraph{links}\label{links}}

The links are a map of arbitrary keys to URLs. In general links to
documentation, source, and a bugtracker are good to have. Additionally
links to the manufacturer/library used are common.

\begin{codefragment}{toml}\noop
[links]  # all optional, arbitrary keys supported
documentation = "https://yaq.fyi/daemons/example-daemon"
source = "https://git.example.com/example-daemon"
bugtracker = "https://git.example.com/example-daemon/-/issues"
\end{codefragment}

\hypertarget{installation-1}{%
\paragraph{installation}\label{installation-1}}

Installation is just like links, except with the specific goal of
pointing to installable package references. For python, this likely
includes a link to PyPI and/or conda-forge.

\begin{codefragment}{toml}\noop
[installation]  # all optional, arbitrary keys supported
PyPI = "https://pypi.org/project/yaqd-example"
conda-forge = "https://anaconda.org/conda-forge/yaqd-example"
\end{codefragment}

\hypertarget{types}{%
\paragraph{types}\label{types}}

All valid
\href{https://avro.apache.org/docs/current/spec.html\#schemas}{Avro
types} are valid for yaq.\\
In addition, \texttt{yaq-tratis} provides a definition for an
N-dimensional homogeneous array, which can be used by putting the string
\texttt{"ndarray"} as the type.\\
Unions of types are specified using TOML arrays.\\
Since TOML does not include a null value (and null is a valid default
value, distinct from not having a default) the string
\texttt{"\_\_null\_\_"} is used in place of the null literal. Note that
when referring to the \emph{type}, \texttt{"null"} is used, just as
\texttt{"int"} is used to specify the integer type.\\
Named types (e.g. records and enums) may be defined inline where they
are used, but may also be provided as an array of tables:

\begin{codefragment}{toml}\noop
[[types]]
name="Greeting"
type="record"
[[types.fields]]
name="message"
type="string"

[[types]]
name="Curse"
type="record"
# You may find inline tables/arrays more readable here
# Note that toml forbids multi-line inline tables (arrays can be multi-line)
fields = [{name="message", "type"="string"}]
\end{codefragment}

\hypertarget{config}{%
\paragraph{config}\label{config}}

Each config parameter is a table with the name of the parameter as the
key within the \texttt{{[}config{]}} table with the following keys:

\begin{description}
\tightlist
\item[\textbf{type} (required)]
Avro type definition, commonly a string such as \texttt{"int"} or
\texttt{"ndarray"}, but may be a table representing collection types or
a record or an array representing a union of types.
\item[\textbf{doc} (optional)]
A string description of the config parameter
\item[\textbf{default} (optional)]
The default value if the config parameter is omitted in the daemon
configuration. If no default is given, it is considered required.
\end{description}

A daemon may override the default value defined by a trait (or provide
one where none was given). Additionally, rather then overwriting the
doc, a key \textbf{addendum} may be added to provide additional context
to the variable (e.g. communicating that a parameter is required despite
a null default value being defined in the trait).

\begin{codefragment}{toml}\noop
[config]

[config.a_binary_config]
type = "boolean"
default = false
doc = "An example of a boolean config param"

[config.an_optional_array]
type = ["null", {type = "array", items = "int"}]
default = "__null__"

# Overriding a config from a trait
[config.baud_rate]
default = 57600
addendum = "I want to provide more info about why 57600 is the default"
\end{codefragment}

\hypertarget{state}{%
\paragraph{state}\label{state}}

The state is configured exactly the same as config, except that
\emph{all} state variable \emph{must} have a default value. The same
rules about overriding defaults and addenda apply to state variables.

\begin{codefragment}{toml}\noop
[state]

[state.a_binary_state]
type = "boolean"
default = true
doc = "An example of a boolean state variable"
\end{codefragment}

\hypertarget{messages}{%
\paragraph{messages}\label{messages}}

Messages are the exposed remote procedure calls over the TCP interface.
This follows closely the specification
\href{https://avro.apache.org/docs/current/spec.html\#Messages}{provided
by Avro}. Note that all errors in the Python implementation are treated
as strings, though that may change in the future.\\
If the request field is omitted, the default is \texttt{{[}{]}}, i.e. no
parameters.\\
If the response field is omitted, the default is \texttt{"null"}.\\
Messages defined by traits should be omitted, only messages unique to
the daemon should be included. Messages should not be overridden, as the
behavior being the same to the client program is the intent of the
traits system.

\begin{codefragment}{toml}\noop
[messages.set_int]
request = [{"name"="int_value", "type"="int"}]
doc = "Set an example int value."

[messages.get_int] response = "int"
doc = "Get the example int value."
\end{codefragment}

\hypertarget{yaq-avpr-file-specification}{%
\subsubsection{yaq AVPR file
specification}\label{yaq-avpr-file-specification}}

The AVPR files generated are compliant with the
\href{https://avro.apache.org/docs/current/spec.html\#Protocol+Declaration}{Avro
specification}, however \texttt{yaq-traits} does add additional
information such that the avpr alone can be used to render the
documentation.\\
The AVPR files include the complete list of \texttt{config} and
\texttt{state} (including those inherited from traits). The links
(including installation links) are also included. Additional keys
specifying which trait provided a method or config/state variable are
also added by \texttt{yaq-traits}


\subsection{\texttt{yaqd-control}}

\hypertarget{installation}{%
\subsubsection{installation}\label{installation}}

yaqd-control can be installed via
\href{https://pypi.org/project/yaqd-control/}{PyPI} or
\href{https://anaconda.org/conda-forge/yaqd-control}{conda-forge}.

\begin{codefragment}{bash}
$ pip install yaqd-control
\end{codefragment}

\begin{codefragment}{bash}
$ conda config --add channels conda-forge
$ conda install yaqd-control
\end{codefragment}

\hypertarget{usage}{%
\subsubsection{usage}\label{usage}}

yaqd-control is a command line application.

Help: learn more, right from your terminal.

\begin{codefragment}{bash}
$ yaqd --help
Usage: yaqd [OPTIONS] COMMAND [ARGS]...

Options:
  --help  Show this message and exit.

Commands:
  clear-cache
  disable
  edit-config
  enable
  list
  reload
  restart
  scan
  start
  status
  stop
\end{codefragment}

Try \texttt{yaqd\ \ -\/-help} to learn more about a particular command.

\hypertarget{the-cache}{%
\paragraph{the cache}\label{the-cache}}

yaqd-control keeps track of known daemons, referred to as the cache

Status: yaqd-control can quickly show you the status of all daemons in
yaqd-control's cache. This is usually the most used subcommand, as it
gives a quick overview of the system, which daemons are offline, and
which are currently busy.

\begin{codefragment}{bash}
$ yaqd status
+-----------+-------+--------------------------+------+---------+-------+
| host      | port  | kind                     | name | status  | busy  |
+-----------+-------+--------------------------+------+---------+-------+
| 127.0.0.1 | 38202 | system-monitor           | foo  | online  | False |
| 127.0.0.1 | 39054 | fake-continuous-hardware | bar  | online  | True  |
| 127.0.0.1 | 39055 | fake-continuous-hardware | baz  | online  | False |
| 127.0.0.1 | 39056 | fake-continuous-hardware | spam | offline | ?     |
| 127.0.0.1 | 37067 | fake-discrete-hardware   | ham  | online  | False |
| 127.0.0.1 | 37066 | fake-discrete-hardware   | eggs | online  | False |
+-----------+-------+--------------------------+------+---------+-------+
\end{codefragment}

List: this is essentially the same as \texttt{status} except that it
does not attempt to contact the daemons, so it does not give you
additional context. List supports a flag -\/-format which accepts "json"
or "toml".

\begin{codefragment}{bash}
$ yaqd list
+-----------+-------+--------------------------+------+
| host      | port  | kind                     | name |
+-----------+-------+--------------------------+------+
| 127.0.0.1 | 38202 | system-monitor           | foo  |
| 127.0.0.1 | 39054 | fake-continuous-hardware | bar  |
| 127.0.0.1 | 39055 | fake-continuous-hardware | baz  |
| 127.0.0.1 | 39056 | fake-continuous-hardware | spam |
| 127.0.0.1 | 37067 | fake-discrete-hardware   | ham  |
| 127.0.0.1 | 37066 | fake-discrete-hardware   | eggs |
+-----------+-------+--------------------------+------+
\end{codefragment}

Scan: Scanning allows you to add currently running daemons to the cache.

\begin{codefragment}{bash}
$ yaqd scan
scanning host 127.0.0.1 from 36000 to 39999...
...saw unchanged daemon fake-discrete-hardware:eggs on port 37066
...saw unchanged daemon fake-discrete-hardware:ham on port 37067
...found new daemon system-monitor:foo on port 38202
...found new daemon fake-continuous-hardware:bar on port 39054
...saw unchanged daemon fake-continuous-hardware:baz on port 39055
...known daemon fake-continuous-hardware:spam on port 39056 not responding
...done!
\end{codefragment}

Scan has some additional options, passed as flags on the command line,
which allow you to change the default scan range and host (for remotely
accessed daemons):

\begin{codefragment}{bash}
$ yaqd scan --help
Usage: yaqd scan [OPTIONS]

Options:
  --host TEXT      Host to scan.
  --start INTEGER  Scan starting point.
  --stop INTEGER   Scan stopping point.
  --help           Show this message and exit.
\end{codefragment}

Edit Config: yaqd-control provides an easy way to edit the default
config file location for a daemon kind. This uses your default editor
(EDITOR environment variable), and defaults to \texttt{notepad.exe} on
Windows, and \texttt{vi} on other platforms. Using yaqd-control to edit
config files means that you do not need to know the default location.
Additionally, it does some basic validity checks (that the toml parses
and that each daemon section has the \texttt{port} keyword). If an error
is found, you are prompted to re-edit the file. Daemons from the config
file are added to the cache. You may pass multiple daemon kinds, which
will be opened in succession.

\begin{codefragment}{bash}
$ yaqd edit-config fake-continuous-hardware system-monitor
\end{codefragment}

Clear Cache: Note that this is a destructive action.
\texttt{clear-cache} deletes all daemons from the cache (thus
\texttt{list} and \texttt{status} will give empty tables) There is no
user feedback.

\begin{codefragment}{bash}
$ yaqd clear-cache
$ yaqd status
+------+------+------+------+--------+------+
| host | port | kind | name | status | busy |
+------+------+------+------+--------+------+
+------+------+------+------+--------+------+
\end{codefragment}

\hypertarget{running-in-the-background}{%
\paragraph{Running in the
background}\label{running-in-the-background}}

Each of the commands in this section can take multiple daemon kinds.

Enable: by enabling a daemon, you allow the operating system to manage
that daemon in the background. An enabled daemon will always start again
when you restart your computer. Enabling is required for the rest of the
commands in this section to work as expected. After enabling, it's
typical to start the daemon as well, this does not happen automatically.
Enablement works in slightly different ways on different platforms, but
the commands are the same (don't worry if the password prompts are
different). Currently supported platforms are Linux (systemd), MacOS
(launchd) and Windows (via NSSM, bundled with the distribution).

\begin{codefragment}{bash}
$ yaqd enable system-monitor
[sudo] password for scipy2020:
==== AUTHENTICATING FOR org.freedesktop.systemd1.manage-unit-files ===
Authentication is required to manage system service or unit files.
Password:
==== AUTHENTICATION COMPLETE ===
\end{codefragment}

Disable: this is the inverse operation to enable, which makes it so that
the daemon does not start on reboot. This does not affect the running
daemon.

\begin{codefragment}{bash}
$ yaqd disable system-monitor
==== AUTHENTICATING FOR org.freedesktop.systemd1.manage-unit-files ===
Authentication is required to manage system service or unit files.
Password:
==== AUTHENTICATION COMPLETE ===
Removed /etc/systemd/system/multi-user.target.wants/yaqd-system-monitor.service.
\end{codefragment}

Start: This starts the daemon running in the background immediately. It
must have been enabled to run in the background using this command.

\begin{codefragment}{bash}
$ yaqd start system-monitor
==== AUTHENTICATING FOR org.freedesktop.systemd1.manage-units ===
Authentication is required to start 'yaqd-system-monitor.service'.
Password:
==== AUTHENTICATION COMPLETE ===
\end{codefragment}

Stop: This stops the daemon running in the background immediately. It
must have been running in the background using yaqd-control (either on
startup via enable or via the start command above).

\begin{codefragment}{bash}
$ yaqd stop system-monitor
==== AUTHENTICATING FOR org.freedesktop.systemd1.manage-units ===
Authentication is required to stop 'yaqd-system-monitor.service'.
Password:
==== AUTHENTICATION COMPLETE ===
\end{codefragment}

Restart/Reload: This stops (if running) and restarts the daemon running
in the background immediately. Reload is slightly different in that it
signals to the daemon to reload its configuration rather than completely
restart, but effectively it is the same as restart (and is a pure alias
where such a signal is not supported). It must have been enabled to run
in the background using this command.

\begin{codefragment}{bash}
$ yaqd restart system-monitor
==== AUTHENTICATING FOR org.freedesktop.systemd1.manage-units ===
Authentication is required to restart 'yaqd-system-monitor.service'.
Password:
==== AUTHENTICATION COMPLETE ===
\end{codefragment}



\subsection{\texttt{yaqc-qtpy}}

\clearpage

\section{Guides}

\subsection{Writing a Daemon}

Intro, reference video of yaqd-new-era

First steps are to communicate with your hardware from python, ignore yaq entirely.
Look at manuals and documentations, look for manufacturer libraries
Reasonable to start interactively, consider making a script

Adding to an existing package or using a cookiecutter

Using cookiecutter
- answer questions
- name your package after manufacturer
- name your daemon

Output of the cookiecutter
- boilerplate
- pyproject.toml
	- defining entry points
- readme
- toml file for daemon
- py file for daemon

Existing package

Writing the toml file
- Consider what traits your daemon will implement
- Add additional messages, properties, types, config, state

Use yaq-traits to convert the toml file into avpr

Implementing the python
Import all of the classes from yaqd_core
Multiple inheritence, order matters

Define methods for each of the messages your protocol supports
Many are inherited via traits, and those traits may have more limited subsets that need to be implemented


Install via flit
Write a config file


\subsection{Configuration versus State}

\subsection{\yaq{} Enhancement Proposals}

\subsection{Implementing traits: has-dependents}

\subsection{Implementing traits: has-limits}

\subsection{Implementing traits: has-mapping}

\subsection{Implementing traits: has-measure-trigger}

\subsection{Implementing traits: has-transformed-position}

\subsection{Implementing traits: has-turret}

\texttt{has-turret} is designed for devices such as monochromators that have a secondary position called a "turret", such as the ones used for selecting gratings.
The trait has three methods, comprising one single \yaq{} property: \texttt{get\_turret}, \texttt{set\_turret}, and \texttt{get\_turret\_options}.
There is also a state value, \texttt{turret}, which stores the current value of the property.

To implement \texttt{has-turret}, the setter and the options getter must be implemented.
The options getter should return a list of string names which are valid options for \texttt{set\_turret}.
The getter is implemented by the \texttt{HasTurret} mix-in class, simply returning the value as stored in the state dictionary.

The state value can either be updated when it is set or asynchronously by reading from the device.

\begin{codefragment}{python}
class MyDaemon(HasTurret, IsDaemon):

    ...

    def get_turret_options(self):
	return ["ir", "vis", "uv"]

    def set_turret(self, turret):
    	self.device.set_turret(self.gratings[turret]["index"])
	self.device.set_position(self._state["destination"])
	self._calculate_limits()
\end{codefragment}

\subsection{Implementing traits: is-homeable}

\texttt{is-homeable} is a trait which provides a single method, \texttt{home}, which must be implemented.

Notably, what ``home'' means in \yaq{} is ``Go to your limit to determine your absolute position, then return to your current destination''.
This is often \textit{not} the same as what an individual device will do when homing, which is often only the first half.
Additionally, homing is a task which takes time to complete, which is contrary to the \yaq{} philosophy which expects methods to return quickly.
As such, there are some common patterns to homing hardware where the \texttt{home} message itself initiates an asynchronous task that actually accomplishes homing.
The details of how each device knows when to move on will vary, but in general it looks something like:

\begin{codefragment}{python}
class MyDaemon(IsHomeable, HasPosition, IsDaemon):

    ...

    def home(self):
        self._busy = True
        self._loop.create_task(self._home())

    async def _home(self):
        self._homing = True
	self._done_homing = False
        self._busy = True
	self.device.home()
	while not self.device.homed():
            await asyncio.sleep(0.01)
        self.set_position(self._state["destination"])
        self._homing = False
\end{codefragment}

Care must be taken to avoid reporting that \texttt{busy} is False at any point during the homing procedure.

\subsection{Implementing traits: is-sensor}

\subsection{Implementing traits: uses-i2c}

\texttt{uses-i2c} is a trait which exists solely to standardize the name of the config value for the I2C address, \texttt{i2c\_addr}.

I2C (inter-integrated circuit) is a low-level communication protocol for serial communication designed for communication between two devices such as microcontollers\cite{}
SMBus\cite{} is a similar protocol that is slightly more strict, but largely is interoperable with I2C, and is thus included for the purposes of the \yaq{} trait.
I2C typically exists on a network with one primary device and a series of secondary devices.
The primary device controls the timing and generates requests for data from the secondary devices.
The address is an integer which is used to designate which secondary device should accept the data and write responses.

Most of the devices that currently exist in the \yaq{} ecosystem which implement this trait are intended to work with a Raspberry Pi, which provides direct access to an I2C bus via the General Purpose Input Output (GPIO) pins.

As I2C is a serial protocol, use of this trait requires the use and implementation of the \texttt{uses-serial} trait.

You can add a default value to your particular configuration if one makes sense for your hardware, though often it is reasonable to say that the address is required in all cases.

\subsection{Implementing traits: uses-serial}

\texttt{uses-serial} is a trait which defines only one message: \texttt{direct\_serial\_write(bytes message)}.
This trait is required by the \texttt{uses-i2c} and \texttt{uses-uart} traits.
The method provided by this trait is intended for use as a debugging tool, and not for normal operations.

To implement a \texttt{uses-serial} daemon, the programmer needs to simply implent the method directly.
You may wish to include logging data read from the device as a response to the command, though what makes sense will depend heavily on the hardware.

\begin{codefragment}{python}
class MyDaemon(UsesSerial, IsDaemon):

    ...

    def direct_serial_write(self, message: bytes) -> None:
        self.ser.write(message)
        out = self.ser.readline()
        self.logger.debug(f"direct serial write: {out.decode()}")
\end{codefragment}


\subsection{Implementing traits: uses-uart}

\texttt{uses-uart} is a trait which exists solely to standardize the names provided to common configuration values among Universal Asynchronous Receiver-Transmitter (UART) style communication.
Common examples of UART style communication are RS-232\cite{} and RS-485\cite{}.

As UART is a serial protocol, use of this trait requires the use and implementation of the \texttt{uses-serial} trait.

These devices will typically appear as a \texttt{COM<n>} port in Windows or something similar to \texttt{/dev/ttyACM0} in Linux.
The operating system specific identifier is the first config field specified by the \texttt{uses-uart} trait, \texttt{serial\_port}.
The second configuration value, \texttt{baud\_rate}, idicates the speed of data transmission, which must be the same for both devices on either end of the transmission.
Some devices have variable baud rates, while others have fixed baud rates.
In the latter case (or even if it is variable, but the device itself has a default) it often makes sense to add a default value for \texttt{baud\_rate} in the TOML file for the protocol.

\begin{codefragment}{toml}\noop
traits = ["uses-uart", "uses-serial", "is-daemon"]
[config]
baud_rate.default = 19200
\end{codefragment}

As a python implementation, the \texttt{UsesUart} class does nothing other than ensure that the required \texttt{UsesSerial} class is included in the class inheritance.
The config values defined by \texttt{uses-uart} are accessed in the normal fashion.

While devices using this trait may wish to consider using some of the more advanced patterns below, it is entirely valid to start with a simple PySerial\cite{} implementation as shown here:

\begin{codefragment}{python}
import serial
from yaqd_core import UsesUart, UsesSerial, IsDaemon

class MyDaemon(UsesUart, UsesSerial, IsDaemon):
    def __init__(self, name, config, config_filepath):
	super().__init__(name, config, config_filepath)
	self._ser = serial.Serial(config["serial_port"], config["baud_rate"])

    def close(self):
	self._ser.close()
\end{codefragment}

\subsection{Daemon patterns: Busy}

\subsection{Daemon patterns: loading and saving state}

\subsection{Daemon patterns: logging}

\subsection{Daemon patterns: serial devices}

\subsection{Daemon patterns: async interfaces}

\clearpage
