\chapter{WALDO: A case study on building a full instrument with \texttt{yaq}} \label{cha:waldo}

\clearpage

\section{Introduction}  % =========================================================================

\clearpage

\section{Design Requirements}  % =============================================================

\subsection{Specifications}

- rep rate
- pulse width
- number of tunable sources
- wavelength regions
- range for delay
- speed of detection, simultaneous acquisition


\subsection{Hardware Selection}

- many from same manufacturers as other systems
   - means yaq daemons are either done, or at least similar enough as to only require minimal perturbations
   - Extant Python interface a plus


\clearpage

\section{Implementing Daemons}  % ===================================================================

Much of the hardware comes before the core upstream laser is on the table
As hardware arrives, daemons can be created _before_ it goes on the table.
   - Some can be surmised from documentation, though it all must be tested thoroughly
   - yaq makes testing the interface easy, as there are limited, well documented, ways to interface with the hardware.
   - Mono had additional features: motorized mirrors and slits
   - Delays were a different model that had some initial bugs
   - Lightcon uses discrete motor positions

   New daemons:
   - Integrated array detector inside of one of the opas
   - gage daemon
      - Less able to be worked out before the system is assembled, as signals are key to ensuring correctness
      - Some done with pulse generators and such, but ultimately hard to fully replicate a running system.
      - discuss chopping

\clearpage

\section{Assemble the Instrument}  % ===================================================================

"The BIG day", when the upstream lasers get installed
   - Done by a tech
   - Can then build around it
      - Chopping
      - Optics for ensuring columation and beam size
      - delays
      - Focusing into sample
      - sample cell itself
      - focusing into mono
      - enclosure
      - alternate beam paths for tuning, etc
Bluesky + yaq = <3
   - Getting to experiments
   - Started with old setup
   - a false start or two with bluesky
Building custom instrument is never "final"
  - New ideas of new experiments to do
  - Temporary setups for short term experiments
     - Yoon group experiments
  - Future addt'l control of existing hardware
     - rep rate
     - chopping
     - upstream laser standby

Include photos and table

\clearpage
