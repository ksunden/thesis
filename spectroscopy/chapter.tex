\chapter{Spectroscopy} \label{cha:spec}

\clearpage

\section{Introduction}  % =========================================================================

Spectroscopy is the study of light interacting with matter.
Light can be manipulated easily and provides a useful analytical tool for interrogating materials.
There are readily available detectors to quantify the amount of light all the way from single photons to powerful bursts.
Simple experiments can be performed with commodity equipment.
The simplest spectroscopic experiments, called linear spectroscopies, are common analytical techniques which are taught to all budding chemists.
Absorbance, fluorescence, reflectance, and raman spectroscopies are the most common techniques.
By observing these four simple techniques, much information about the quantum states of a material can be obtained.
Different techniques are sensitive to modes with different characteristics of the transitions, observing different selection rules.

Nonlinear techniques expand on the available information by interrogating coupling between multiple states.
This allows for resolution of congested spectra \cite{},  and performing experiments which rely on states that would otherwise be disallowed by selection rules\cite{}.
These techniques, broadly categorized as Coherent Multidimensional Spectroscopy (CMDS), have many parallels to common NMR measurements\cite{}.
In the same ways that multiple NMR pulses reveal coupling between nuclear spin states, so do multiple frequencies of light reveal coupling between optical and vibrational states of molecules.
CMDS relies on creating coherences, or superpositions, of multiple states.
For this reason, they are sometimes more colloquially referred to as ``Shr\"odinger cat state spectroscopy''\cite{}.

Most CMDS experiments are performed in the time domain, using broadband pulses which are used to create interferograms by scanning in time\cite{}.
This is similar in principle to Fourier Transform infrared absorbance spectroscopy.
Time domain CMDS experiments evolved from the tradition of pump-probe and photon echo spectroscopies\cite{}.
Time domain experiments are limited by the pulse bandwidth of a single excitation pulse\cite{}.
While much progress has been made in reliably producing broadband light sources\cite{}, the available range remains small in comparison to the range of many electronic and vibrational states.
Additionally, the heterodyne detection of time domain techniques requires a local oscillator to produce the interferometric measurement.
This limits the mixing processes available for interrogation, as an output color equal to one of the input colors is required.

Frequency domain approaches
 - various names, multi resonant, floquet state
 - no need for broadband source
 - homodyne detection
 - matching FID to coherence lifetimes
 - Use multiple light sources, generate new light at sum or difference frequencies
   - "I do math with light"
 - Raman and CARS

Challenges
 artifacts: absorptive, pulse effects, window effects, group/phase velocity mismatch
 many moving parts

Experimentally, a frequency domain instrument requires two or more tunable light sources.
These light sources are typically Optical Parametric Amplifiers (OPAs) or Optical Parametric Oscillators (OPOs).
Additionally, the frequency domain experiment requires a controllable time delay to both ensure temporal overlap and act as a discriminating axis for experiments.
The delays are not used in the same fashion as time domain experiments, but are still a crucial part of the instrument.
All of the incident laser pulses must arrive at the sample overlapped in time and space, and high intensities are required to elicit nonlinear response, so focusing optics are important.
Introducing a small time delay between successive pulses can often greatly increase signal to noise ratios because non-resonant background signal is diminished to a greater degree than the resonant material response.

\clearpage
